\chapter{Introduction}


Radars have several highly attractive properties: they are active systems independent on lighting conditions, are very fast and have high precision. Furthermore, implemented at millimeter-wave wavelengths, radar sensors can be designed as low-power devices with no moving parts in a favorable form factor \citep{lien_gillian_karagozler_amihood_schwesig_olson_raja_poupyrev_2016}.

Radar technology has been around since the 1930s \citep{watson-watt_1945}, and has since developed into a well established field of engineering. Although the archetypal use case in radio-frequency (\gls{rf}), radars regard detection and tracking of large objects at far distances, such as for air and marine monitoring, new areas of applications have emerged under recent years posing very different engineering challenges. Some of these are outlined in \citep{amin_2017}, where close-range radars are used for applications such as vital signs monitoring \citep{kuo_lin_yu_lo_lyu_chou_chuang_2016}, gesture recognition \citep{lien_gillian_karagozler_amihood_schwesig_olson_raja_poupyrev_2016} and tumour imaging \citep{klemm_gibbins_leendertz_horseman_preece_benjamin_craddock_2011} to name a few. Furthermore, millimeter-wave radars have lately become more inexpensive, largely due to their widespread adoption in the automotive industry \citep{frenzel_2018}, making such devices an attractive option in a wide range of low cost applications.

Alongside the miniaturization of radar systems a trend in data science has emerged; the movement towards machine learning centric methods. This transition, spanned over multiple decades, has placed machine learning as an increasingly popular choice for data analysis. Machine learning algorithms encompass a very large set of algorithms, but fundamentally utilize various statistical techniques in order to allow computer systems to progressively improve performance \citep{a_smola_svn_vishwanathan_2010}. Data scientists have found machine learning, which forms a subclass of aritficial intelligence, particularly practical for data that lack an easily predictable structure.

In this work we investigate whether millimeter-wave radar can be used for surface recognition. Two broad antenna beams illuminate a target scene, and using a feature extraction procedure and machine learning techniques the returning echo is analyzed in order to classify the surface type. This work focuses on the use case of determining if a surface is grassy or not, but could easily be extended to invoke other surface types as well.  

\section{Motivation and previous work}

Autonomous robots have found increasing use in numerous devices, from helping customers navigate stores \citep{mcsweeney_2018} to keeping floors clean \citep{sanfacon_2017} and mowing lawns \citep{udelhofen_2018}. A common challenge with such systems is keeping the robot in bounds. This commonly means for the robot to be aware where on a two-dimensional map it is currently located. In certain applications however, "in bounds" rather involves staying on one type of material, such as autonomous robot lawn mowers staying only on areas covered in grass, or robot vacuum cleaners on regions without a carpet. In such devices one may be content with knowing that the robot roams around remaining on its designated surface type rather than having knowledge of its exact position. 

Surface classification can also be used in autonomous devices as a supporting system. A robot vacuum cleaner could for instance make use of such a system in numerous ways, such as for avoiding liquid spills or using surface-dependant cleaning programs. One could easily imagine other use cases where such a system would be a great convenience.

In order to distinguish one surface type from another some feature of the surface at hand must somehow be captured and analyzed. This very general problem statement can and has been approached from many different angles.

Taking inspiration from the recent advances in computer vision, one may be tempted to use cameras for visual inspection of the surface at hand. Computer vision has indeed attracted extensive research over the past few decades with impressive results \citep{liu_chen_fieguth_zhao_chellappa_pietikäinen_2018}. In a computer vision framework, images of different surfaces can be separated by their differences in \emph{texture} - their spatial organization of basic elements. Such fundamental microstructures obey some kind of statistical properties which can be percepted by for instance a convolutional neural network. Effective texture identification of textures in image databases was used in, for example, \citep{do_vetterli_2002} for accuracte classification. 

Although using computer vision for accurate real time target identification is an attractive option, it is commonly an infeasible option for small devices with limited hardware capabilities. Cameras capture light in the visible frequency spectrum, which inherently renders them sensitive to changes in light conditions. Thus, unless direct illumination of the target surface is used, solutions involving cameras are highly dependent on ambient light. Such a limitation can make cameras impractical for small devices navigating areas with varying lighting.

The perhaps most immediate way to perform surface identification is through direct contact. In \citep{giguere_dudek_2011}, surface identification for low-velocity mobile robots using a small metallic rod with an attached accelerometer. By capturing accelerometer output induced in the tip of the rod during robot motion identification was possible for a couple of different surface types. While probing may produce appealing results in certain situations, the method is fundamentally based on physical contact with the target. Hence, a probe is more susceptible to damage, can more easily get stuck and is more exposed to detrimental tear over time than its non-interfering counterparts.

% Two chief methods for doing this - Ultrsonic and radar. 
For surface classification with devices in motion, the seemingly two chief non-contact methods researchers have spent their time on for this application are ultrasonic methods and radar-based methods. These two methods are both noninvasive and have modest power requirements. Furthermore, both are active systems and thus do not need to rely on any ambient signal source. Ultrasonic and radar systems have been tested by researchers for road condition monitoring \citep{bystrov_2016}, \citep{mckerrow_kristiansen_2006}. 

These two methods primarily utilize the difference in roughness, a geometric characteristic of a surface caused by spatial variations in surface depth. Researchers experimenting with ultrasonic sensors (perhaps motivated by a study proving bats capable of discriminating surfaces with different roughness with their echolocation \citep{schmidt_1988}) have found some degree of success. In \citep{politis_probert_1999} angled ultrasonic sensors were used to measure roughness. By comparing the distribution of energy between specular (i.e. mirror-like) and diffuse componenets of ultrasonic echo they were able to distinguish between six different surface types, and modeling the floor as having a random structure on the form of a Gaussian distribution around the surface plane creating. Using this method the researchers were able to generate predictions on a sample-by-sample basis, not taking the temporal dimension into account.
% Work with the last part here - radar applications/radarCat etc. We wish to utilize time-based methods. Would be neat to find radar work doing nearly the same thing. Finally - we are using a single recieving antenna.

\begin{figure}
	\centering
	\includegraphics[scale=0.8]{figs_temp/acc_sensor}
	\caption{Image of the 60 GHz PCR radar sensor used in this work. Image retrieved from \emph{www.acconeer.com/products}.}
	\label{fig:acc_sens}
\end{figure}


\section{Objective}

In this work, we seek to create a model capable of performing binary surface classification from data collected during robot movement. By collecting data on grass surfaces and selection of non-grass surfaces we wish to create an effective classifier able to distinguish a grass covered surface from other surface types. All experiments will be performed using two 60 Ghz pulsed coherent radar \gls{pcr} sensors from Acconeer, see figure \ref{fig:acc_sens},  mounted on the front of device moving straight at a constant velocity. 

\section{Thesis outline}

The oulline of this thesis is as follows:
\\ \\
\noindent\textbf{Chapter 2:} This chapter explains the fundamentals of \gls{pcr} systems, as well as describe how distance, reflectivity and radial velocity can be extracted from the radar response. 
\\ \\
\noindent\textbf{Chapter 3:} In chapter 3 we proceed with describing the method used for collecting the data used in this project. Sensor settings, such as pulse length and sampling frequency, are discussed. 
\\ \\
\noindent\textbf{Chapter 4:} Chapter 4 presents the feature extraction process for surface identification. Temporal features, such as the autocorrelation function and the fourier transform, are investigated for accurate surface classification. At the end of the chapter, one of the tested extraction methods is selected for further investigation. 
\\ \\
\noindent\textbf{Chapter 5:}  In this chapter, some different classfiers are tested for prediction accuracy. Both linear and non-linear machine learning models are tested. At the end of the chapter, one machine learning classifier is found to be the best. 
\\ \\
\noindent\textbf{Chapter 6:} In chapter 6 some postprocessing and optimizations to the best found model are discussed. 
\\ \\
\noindent\textbf{Chapter 7:} Chapter 7 examines the found models capabilities in real and artificially created test scenarios. The found results are discussed, and some limitations are discussed.
\\ \\
\noindent\textbf{Chapter 8:} In chapter 8 conclusions of this work is presented and possible directions of future work is discussed. 







%In spite of their rise in populatity, solutions for surface classification using high frequency radars are somewhat scarce. When a single radar sensor is used for this purpose and a nonstationary setting is considered such as for a device moving across a surface of interest, very little previous work can be found in litterature. The perhaps closest resemblance is the RadarCat project\citep{yeo_2016}. As a part of Google's project Soli, RadarCat used a radar sensor for material classification. Nonetheless, a central part of this approach was having an object stationary and in direct contact with the radar sensor. The focal point of Project Soli was gesture recognition. This application bears some similarities to surface classification in motion in the regard that the subject is nonstationary. However classifying surfaces during motion and classifying hand gestures differ on a critical point - gestures are actions performed during some window in time while a robot moving along a surface is a continuous operation. 



%Continue: You want to be able to do this on a budget.


%\subsection{Previous work}


%It is in many applications of interest to obtain images of subsurface identification nondestructively. In recent decades, ground penetrating radar (GPR) has found increasing use in evaluation of road conditions \citep{solla_gonzález-jorge_varela_lorenzo_2013} for preservation and maintenance of infrastructure.


%A major challenge in radar sensing is to not only detect, but also to identify radar targets. This can for example be used for monitoring of urban environments \citep{harter_kowalewski_sit_jalilvand_ziroff_zwick_2014}.


%Localization is the classic use-case for radars. 
%\\ \\
% Radar  can detect relatively small targets at near or far distances and can measure their range with  precision  in  all weather,  which is  its chief  advantage when compared with other  sensors \citep{skolnik_2009}
%\\ \\
%Radar classification: 


% Classifying underground objects - this is a good paper we should look deeper into
%Using a ground penetrating radar (GPR), it is possible to identify subsurface objects. 
%Significantly lower frequency (900 MHz)
%\citep{lu_pu_liu_2014}.
% Paper on ground penetrating radar
%\citep{daniels_2004}

% RadarCat


%Chapter 3: Mention something about the data collecting process. Motive our choice of parameters such as sampling frequency. Go through visual observations of our data.

%Chapter 4: Feature selection and preprocessing is considered. Signal structure is discussed primarily on an intuitive level. 

%Chapter 5: Classification schemes. Moving from simple to advanced, some classifiers of special interest are investigated. Results with regards to these classifiers are presented.

%Chapter 6: Disucssion.

%Chapter 7: Conclusion.
