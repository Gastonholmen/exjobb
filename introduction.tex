\chapter{Introduction}

This paper presents an algorithm for surface classification using millimeter-wave radar. 

Fundamentally, radar is based on the concept of transmitting and recieving electromagnetic radiation. Radar response depends on the surrounding region of the sensor as objects in its vicinity scatter transmitted waves differently \citep{richards_2014}.  

 As technolgy is becoming increasingly intertwined with everyday life a multitude of new areas for smart sensing has opened up. One sensor category that has been recieving much attention is radars with very high frequency. These sensors combine a multitude of desirable features packed into a favorable form factor. These capabilities has made high frequency radars an increasingly popular choice in a steadily growing number of applications, such as monitoring of vital signs \citep{kuo_lin_yu_lo_lyu_chou_chuang_2016}, gesture recognition \citep{lien_gillian_karagozler_amihood_schwesig_olson_raja_poupyrev_2016}

Alongside with this development another trend within data science has emerged; the movement towards machine learning centric methods . This transition, spanned over multiple decades, has placed machine learning as one of the main-stays of information technology and a rather central part of our life. With ever increasing amounts of data, smart data analysis has proven key for technological progress \citep{a_smola_svn_vishwanathan_2010}.

Machine learning algorithms fundamentally utilizes various statistical techniques in order to allow computer systems to progressively improve performance. Data scientists have found this subclass of aritficial intelligence extremely practical, especially for data that lack an easily predictable structure. 

The amount of millimetre-wave radar applications is growing steadily.

With a continuously increasing demand for small sensors for use in a world where 

\section{Motivation}

Classification of surfaces is an important task which appears in many areas. There are already numerous methods for classifying surfaces which are based on, for instance, optics and probing. As for optics, using images has been a successful approach in determining surface roughness. One such method involves analyzing images of three-dimensional surface textures captured with different directions of illumination. After going through image processing techniques, the data was fed through a support vector machine for classification \citep{dong_duan_yang_2008}. 

For probing on the other hand one resorts to measuring directly onto a surface through direct contact. In \citep{song_han_hu_li_2014} probing is used by sweeping a thin film across a fabric with constant contact force a depth-dependant charge is induced. After reading the output charge over some time interval it is possible to extract information for detecting the texture of the fabric. A similar approach was used in \citep{strese_schuwerk_iepure_steinbach_2017}, where again surface classification was performed through direct contact. In this case features where instead extracted from the sound produced by vibrations generated by the movement. 

Both of the approaches above have drawbacks that make them unsuited for this work. The optical approach requires a decent amount of light in order to work. This requirement could potentially limit the applications of the result from this work. Using a radar based solution instead, we can perform surface classification without having to rely on ambient light, nor having to emit any visible light.

When it comes to probing, it is necessary that the measuring instruments touch the underlying surface. It has been proven that probing can be used successfully in surface identification of rough terrain in \citep{giguere_dudek_2011}. At the end of a small metal rod, an accelerometer is attached, which measures acceleration patterns that are later used for classification. However, with an instrument that has direct contact with the ground, the chances that the instruments breaks, or that it drags other objects with it increases significantly. Hence our radar based approach has an advantage over probing in this remark.

optical, probing etc.
\\ \\
\noindent Optical: Images has been successful in determining surface roughness. One such method involves images of three-dimensional surface textures under different directions of illumination, and after some image processing techniques using a support vector machine for classification \citep{dong_duan_yang_2008}. 
\\ \\
\noindent Probing: One may also measure directly onto a surface directly through a probing approach. By sweeping a thin film across a fabric with constant contact force a depth-dependant charge is induced. After reading the output charge over some time interval it is then possible to extract information for detecting the texture of the fabric \citep{song_han_hu_li_2014}. A similar approach was used in \citep{strese_schuwerk_iepure_steinbach_2017}, where again surface classification was performed through direct contact. In this case features where instead extracted from the sound produced by vibrations generated by the movement. 
\\ \\
It is often desirable to to this without direct contact i.e. probing
\\ \\
Localization is the classic use-case for radars. 
\\ \\
 Radar  can detect relatively small targets at near or far distances and can measure their range with  precision  in  all weather,  which is  its chief  advantage when compared with other  sensors \citep{skolnik}
\\ \\
Radar classification: Previous work and how they are different: A major challenge in radar sensing is to not only detect, but also to identify radar targets. This can for example be used for monitoring of urban environments \citep{harter_kowalewski_sit_jalilvand_ziroff_zwick_2014}, or with a radar system mounted on a rotary table unit. 

 of urban environments and automotive radar 
\\ \\
Applications involving millimetre-wave radars and surface recognition is scarce.  
\\ \\
These radars commonly have wavelenghts of something something
\\ \\
If we however want to classify surface materials using a much shorter wavelength, the task changes dramatically as apsorption is near-instant. 
\\ \\
Something about radar wavelenght categories
\\ \\
Furthermore, radar sensors are commonly used in an array setting which permits beamforming and extraction of spatial information. 
\\ \\
If one on the other hand only has a singular radar sensor this is not possible and you must resort to other means.
\\ \\
Probably want something about IQ demodulation somewhere in here.

\section{Problem formulation}
With this work, we intend to present a solution towards surface identification based on radar data. The data is collected using two 60 GHz Acconeer radar sensors, each with a sampling frequency of 200 Hz. Both sensors are mounted at the front of a robot, and are assumed to move across a surface with a constant speed and height.

Once obtained, the data will undergo feature extraction. Ideally, we want few features that carry as much information as possible. The next step is to develop a machine learning model into which these features will be feeded. Several different model types will be investigated and evaluated, after which we choose one to proceed with and optimize. 

The working procedure is illustrated in figure (....)

%We develop a pipeline for effectively solving this problem. 
%\\ \\
%We present modelling options with high accuracy and efficacy. 
%\\ \\ 
%We present results based on a 60GHz Acconeer micro radar. 
%\\ \\
%In this paper we introduce means of surface identification based on the output from a single 60 GHz radar sensor moving across a surface with constant height. 

\section{Thesis outline}

In chapter 2 we present an overview of a general radar system and introduce some crucial concepts for understanding the radar output used in this work. In chapter 3 we proceed by presenting our measurement setup and comment on the data collecting process and the data itself. Chapter 4 revolves around discussing ways of preprocessing the data and extracting relevant features from it. After the preprocessing, in chapter 5, we go through different classification schemes we have tested. The classification results are presented in a table for easy comparison. Finally, we conclude with an overall discussion of the work in chapter 6 and summarize our conclusions in chapter 7.

%Chapter 3: Mention something about the data collecting process. Motive our choice of parameters such as sampling frequency. Go through visual observations of our data.

%Chapter 4: Feature selection and preprocessing is considered. Signal structure is discussed primarily on an intuitive level. 

%Chapter 5: Classification schemes. Moving from simple to advanced, some classifiers of special interest are investigated. Results with regards to these classifiers are presented.

%Chapter 6: Disucssion.

%Chapter 7: Conclusion.
