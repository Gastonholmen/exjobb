\chapter{Introduction}


Radars have several highly attractive properties: they are active systems independent on lighting conditions, are very fast and have high precision. Furthermore, implemented at millimeter-wave RF frequencies, a radar sensor can be designed as a low-power device with no moving parts and a favorable form factor\citep{lien_gillian_karagozler_amihood_schwesig_olson_raja_poupyrev_2016}.

Radar technology has been around since the 1930s \citep{watson-watt_1945}, and has since developed into a well established field of engineering. Although the archetypal use case in RF radars regard detection and tracking of large objects at far distances, such as for air and marine monitoring, new areas of applications have emerged under recent years posing very different engineering challenges. Some of these are outlined in \citep{amin_2017}, where close-range radars are used vital signs monitoring \citep{kuo_lin_yu_lo_lyu_chou_chuang_2016}, gesture recognition \citep{lien_gillian_karagozler_amihood_schwesig_olson_raja_poupyrev_2016} and tumour imaging \citep{klemm_gibbins_leendertz_horseman_preece_benjamin_craddock_2011} to name a few. Furthermore, millimeter-wave radars have lately become more inexpensive, largely due to their widespread adoption in the automotive industry \citep{frenzel_2018}.

Alongside the miniaturization of radar systems a trend in data science has emerged; the movement towards machine learning centric methods. This transition, spanned over multiple decades, has placed machine learning as one of the main-stays of information technology. Machine learning algorithms encompass a very large set of algorithms, but fundamentally utilize various statistical techniques in order to allow computer systems to progressively improve performance \citep{a_smola_svn_vishwanathan_2010}. With ever increasing amounts of data, smart data analysis has proven key for technological progress . Data scientists have found machine learning, which forms a subclass of aritficial intelligence, extremely practical especially for data that lack an easily predictable structure.

In this work we investigate whether millimeter-wave radar can be used for surface recognition. We propose an approach using two non-synchronized 60 GHz radar sensors for surface classification. Two broad antenna beams illuminate a scene capturing a complex refelected signal, and using signal processing and machine learning techniques we classify the surface type. This work focuses on the use case of determining if a surface is grassy or not, but could easily be extended to invoke other surface types as well.  

\section{Motivation and previous work}

Autonomous robots have found increasing use in numerous systems, from helping customers navigate stores \citep{mcsweeney_2018} to keeping floors clean \citep{ifr_press_release_2016} and mowing lawns \citep{udelhofen_2018}. A common challenge with such systems is keeping the robot in bounds. This commonly means for the robot to be aware where on a two-dimensional map it is currently located. In certain applications however, "in bounds" rather involves staying on one type of material, such as autonomous robot lawn mowers staying only on areas covered in grass, or robot vacuum cleaners on regions without a carpet. In such devices one may be content with knowing that the robot roams around remaining on its designated surface type rather than having knowledge of its exact position. 

Surface classification can also be used in autonomous devices as a supporting system. For the example used above, a robot vacuum cleaner could make use of such a system in numerous ways, such as for avoiding liquid spills and using surface-dependant cleaning programs. One could easily imagine other use cases where such a system would be a great convenience. Thus, it would be immensely useful to devise a system able to determine surface type. 

In order to distinguish one surface type from another some feature of the surface at hand must somehow be captured and analyzed. This very general problem statement can and has been approached from many different angles. 

Taking inspiration from the recent advances in autonomous vehicles, one may be tempted to use cameras for visual inspection of the surface at hand. Computer vision has indeed attracted extensive research over the past few decades with impressive results \citep{liu_chen_fieguth_zhao_chellappa_pietikäinen_2018}. In a computer vision framework, images of different surfaces can be separated by their differences in \emph{texture} - their spatial organization of basic elements. Such fundamental microstructures obey some kind of statistical properties which can be percepted by for instance a convolutional neural network. Effective texture identification of textures in image databases was used in, for example, \citep{do_vetterli_2002} for accuracte classification. 

Although using computer vision for accurate real time target identification is an attractive option, it is commonly an infeasible option for small devices with limited hardware capabilities. Cameras capture light in the visible frequency spectrum, which inherently renders them sensitive to changes in light conditions. Thus, unless direct illumination of the target surface is used, solutions involving cameras are highly dependent on ambient light. Such a limitation can make cameras impractical for small devices navigating areas with varying lighting. 

The perhaps most immediate way to perform surface identification is through direct contact. In \citep{giguere_dudek_2011}, surface identification for low-velocity mobile robots considered using a small metallic rod with an attached accelerometer. By capturing accelerometer output induced in the tip of the rod during robot motion identification was possible for a couple of different surface types. While probing may produce appealing results in certain situations, the method is fundamentally based on physical contact with the target. Hence, a probe is more susceptible to damage, can more easily get stuck and is more exposed to detrimental tear over time than its non-interfering counterparts. 

% Two chief methods for doing this - Ultrsonic and radar. 
The seemingly two chief methods researchers have spent their time on for this application are ultrasonic methods and radar-based methods. These two methods are both noninvasive and have modest power requirements. Being active systems, meaning that they do not require any ambient signal source. Ultrasonic and radar systems have been tested by researchers for road condition monitoring \citep{bystrov_2016}, \citep{mckerrow_kristiansen_2006}.

\section{Radar Fundamentals}

Fundamentally, a radar operates by radiating radio frequency (RF) electromagnetic energy and listening if the transmitted energy generates any echoes \citep{skolnik_2009}. By analyzing properties of the returning signal it is then possible to obtain information about the target objects that scattered the transmitted pulse. This may involve the distance and angle at which scatterers are located or which velocity they are moving in. With a sufficiently high angular and range resolution it is possible to descern parts of the ragets' sizes and shapes.  

If a perfect scatterer is located a distance $d$ from a radar transmitter, and a radar pulse with envelope $A(t)$ and angular frequency $\Omega$ is transmitted towards it, the returning signal will have the form \citep{richards_2014}

\begin{equation}
	r(t) = CA(t)\sin(\Omega t + \theta(t))
\end{equation}

where $C$ is some constant related to the dissapation of power throughout space and some radar system properties. % More on this


\section{Thesis outline}

In chapter 2 we present an overview of a general radar system and introduce some crucial concepts for understanding the radar output used in this work. In chapter 3 we proceed by presenting our measurement setup and comment on the data collecting process and the data itself. Chapter 4 revolves around discussing ways of preprocessing the data and extracting relevant features from it. After the preprocessing, in chapter 5, we go through different classification schemes we have tested. The classification results are presented in a table for easy comparison. Finally, we conclude with an overall discussion of the work in chapter 6 and summarize our conclusions in chapter 7.






%In spite of their rise in populatity, solutions for surface classification using high frequency radars are somewhat scarce. When a single radar sensor is used for this purpose and a nonstationary setting is considered such as for a device moving across a surface of interest, very little previous work can be found in litterature. The perhaps closest resemblance is the RadarCat project\citep{yeo_2016}. As a part of Google's project Soli, RadarCat used a radar sensor for material classification. Nonetheless, a central part of this approach was having an object stationary and in direct contact with the radar sensor. The focal point of Project Soli was gesture recognition. This application bears some similarities to surface classification in motion in the regard that the subject is nonstationary. However classifying surfaces during motion and classifying hand gestures differ on a critical point - gestures are actions performed during some window in time while a robot moving along a surface is a continuous operation. 



%Continue: You want to be able to do this on a budget.

%\subsection{Problem formulation}
%With this work, we intend to present a solution towards surface identification based on radar data. The data is collected using two 60 GHz Acconeer radar sensors, each with a sampling frequency of 200 Hz. Both sensors are mounted at the front of a robot, and are assumed to move across a surface with a constant speed and height.

%Once obtained, the data will undergo feature extraction. Ideally, we want few features that carry as much information as possible. The next step is to develop a machine learning model into which these features will be feeded. Several different model types will be investigated and evaluated, after which we choose one to proceed with and optimize. 

%The working procedure is illustrated in figure (....)

%We develop a pipeline for effectively solving this problem. 
%We present modelling options with high accuracy and efficacy. 
%We present results based on a 60GHz Acconeer micro radar. 
%In this paper we introduce means of surface identification based on the output from a single 60 GHz radar sensor moving across a surface with constant height. 


%\subsection{Previous work}


%It is in many applications of interest to obtain images of subsurface identification nondestructively. In recent decades, ground penetrating radar (GPR) has found increasing use in evaluation of road conditions \citep{solla_gonzález-jorge_varela_lorenzo_2013} for preservation and maintenance of infrastructure.


%A major challenge in radar sensing is to not only detect, but also to identify radar targets. This can for example be used for monitoring of urban environments \citep{harter_kowalewski_sit_jalilvand_ziroff_zwick_2014}.


%Localization is the classic use-case for radars. 
%\\ \\
% Radar  can detect relatively small targets at near or far distances and can measure their range with  precision  in  all weather,  which is  its chief  advantage when compared with other  sensors \citep{skolnik_2009}
%\\ \\
%Radar classification: 


% Classifying underground objects - this is a good paper we should look deeper into
%Using a ground penetrating radar (GPR), it is possible to identify subsurface objects. 
%Significantly lower frequency (900 MHz)
%\citep{lu_pu_liu_2014}.
% Paper on ground penetrating radar
%\citep{daniels_2004}

% RadarCat


%Chapter 3: Mention something about the data collecting process. Motive our choice of parameters such as sampling frequency. Go through visual observations of our data.

%Chapter 4: Feature selection and preprocessing is considered. Signal structure is discussed primarily on an intuitive level. 

%Chapter 5: Classification schemes. Moving from simple to advanced, some classifiers of special interest are investigated. Results with regards to these classifiers are presented.

%Chapter 6: Disucssion.

%Chapter 7: Conclusion.
