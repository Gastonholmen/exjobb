\chapter{Introduction}

\section{Smart sensing and machine learning}

As technology is becoming increasingly intertwined with everyday life, smart sensing has become an integral component in many areas of engineering. From watches to cars, applications where no or only rudimentary sensoring equipment previously were utilized now hold a multitude of advanced sensors. Many business and technical leades have articulated that a world where an abundance of sensors is emerging, largely propelled by the Internet of Things \citep{bryzek_2013}. 

One sensor category that recently has been recieving much attention is high frequency radars \citep{frenzel_2018}. Finding use in the automotive industry, radars have been adopted in a wide variety of systems, such as adaptive cruise control, automatic breaking and backup object detection. This development has greatly impacted the sales volume, subsequently lowering the price and making millimeter-wave radars attractive in other applications. These radar systems can be packed into a favorable form factor, which further facilitates development. These capabilities has made high frequency radars an increasingly popular choice in a steadily growing number of applications, such as monitoring of vital signs \citep{kuo_lin_yu_lo_lyu_chou_chuang_2016}, gesture recognition \citep{lien_gillian_karagozler_amihood_schwesig_olson_raja_poupyrev_2016}.

Alongside with this development another trend in data science has emerged; the movement towards machine learning centric methods . This transition, spanned over multiple decades, has placed machine learning as one of the main-stays of information technology and a rather central part of our life. With ever increasing amounts of data, smart data analysis has proven key for technological progress \citep{a_smola_svn_vishwanathan_2010}.

Machine learning algorithms fundamentally utilize various statistical techniques in order to allow computer systems to progressively improve performance. Data scientists have found this subclass of aritficial intelligence extremely practical, especially for data that lack an easily predictable structure. 

% One more paragraph here more closely related to this idea

\section{Motivation}

Autonomous robots have found increasing use, from helping customers navigate stores \citep{mcsweeney_2018} to keeping floors clean [reference needed] and mowing lawns \citep{udelhofen_2018}. A common challenge with such systems is keeping the robot in bounds. This commonly means for the robot to be aware where on a two-dimensional map it is currently located, achieved for instance through equipping the robot with ultrasonic transmitters and finding naturally occurring "beacons" for localization \citep{leonard_durrant-whyte_1991}. In certain applications however, "in bounds" rather involves staying on one type of material, such as autonomous robot lawn mowers staying only on areas covered in grass, or robot vacuum cleaners on regions without a carpet. In such devices one may be content with knowing that the robot roams around remaining on its designated surface type rather than knowing its exact position. 

Surface classification can also be used in autonomous devices as a supporting system. For the example used above, a robot vacuum cleaner could make use of such a system in numerous ways, such as for avoiding liquid spills and using surface-dependant cleaning programs. One could easily imagine other use cases where this  would be a great convenience. Thus, it would be immensely useful to devise a system able to determine surface type. Surface classification, in its most general form, is a common problem reappearing in many areas and has thus naturally been approached from many different angles. 

Taking inspiration from the recent advances in autonomous vehicles, one may be tempted to use cameras for visual inspection of the surface at hand. Computer vision has indeed attracted extensive research over the past few decades with impressive results \citep{liu_chen_fieguth_zhao_chellappa_pietikäinen_2018}. In a computer vision framework, images of different surfaces can be separated by their differences in \emph{texture} - their spatial organization of basic elements. Such fundamental microstructures obey some kind of statistical properties which can be percepted by for instance a convolutional neural network (CNN). Effective texture identification of textures in image databases was used in, for example, \citep{do_vetterli_2002} for accuracte classification.

Cameras capture light in the visible frequency spectrum, which inherently renders them sensitive to changes in light conditions. Unless direct illumination of the target surface is used, a solution involving a camera is a passive device dependent on ambient light. Such a limitation can make cameras unsuitable for surface classification where light conditions varies and low power consumption is critical. Furthermore, making continuous use of CNNs is a computationally expensive operation potentially impractical on small devices with limited hardware capabilities. 

The perhaps most immediate way to circumvent the problem of light conditions is through a probing approach. Here, one resorts to measuring onto a surface through direct contact. In \citep{song_han_hu_li_2014} probing is used by sweeping a thin film across a fabric with constant contact force. After reading a depth-dependant induced output charge over some time interval it was then possible to extract information for detecting the texture of the fabric. A similar approach was used in \citep{strese_schuwerk_iepure_steinbach_2017}, where again surface classification was performed through direct contact. In this case features where instead extracted from sound produced by vibrations generated by movement across the texture. In \citep{giguere_dudek_2011}, surface identification for low-velocity mobile robots is considered using a small metallic rod with an attached accelerometer. By capturing accelerometer output induced in the tip of this rod during robot motion identification was possible for a couple different surface types. While probing may produce appealing results in certain situations, they are fundamentally based on physical contact with the target of interest. Hence, a probe is more susceptible to damage, can more easily get stuck and is more exposed to detrimental tear over time than its non-interfering counterparts. 

ULTRASONIC: COMPARE ULTRASONIC TO RADAR	
Ultrasonic sensors have been used in many studies for real-time identification of road surfaces. 
In 

Short-range ultrasonic sensors where used in \citep{bystrov_2016} for identification of road surface conditions. \citep{mckerrow_kristiansen_2006} 

In spite of their rise in populatity, solutions using high frequency radars are scarce. When a single radar sensor is used for this purpose and a nonstationary setting is considered such as for a device moving across a surface of interest, seemingly no previous work can be found in litterature. The perhaps closest resemblance is the RadarCat project\citep{yeo_2016}. As a part of Google's project Soli, RadarCat used a radar sensor for material classification. Nonetheless, a central part of this approach was having an object stationary and in direct contact with the radar sensor. The focal point of Project Soli was gesture recognition. This application bears some similarities to surface classification in motion in the regard that the subject is nonstationary. However classifying surfaces during motion and classifying hand gestures differ on a critical point - gestures are actions performed during some window in time while a robot moving along a surface is a continuous operation. 



Continue: You want to be able to do this on a budget.

\subsection{Problem formulation}
With this work, we intend to present a solution towards surface identification based on radar data. The data is collected using two 60 GHz Acconeer radar sensors, each with a sampling frequency of 200 Hz. Both sensors are mounted at the front of a robot, and are assumed to move across a surface with a constant speed and height.

Once obtained, the data will undergo feature extraction. Ideally, we want few features that carry as much information as possible. The next step is to develop a machine learning model into which these features will be feeded. Several different model types will be investigated and evaluated, after which we choose one to proceed with and optimize. 

The working procedure is illustrated in figure (....)

%We develop a pipeline for effectively solving this problem. 
%We present modelling options with high accuracy and efficacy. 
%We present results based on a 60GHz Acconeer micro radar. 
%In this paper we introduce means of surface identification based on the output from a single 60 GHz radar sensor moving across a surface with constant height. 


\subsection{Previous work}

Surface identification is a common problem reappearing in many areas, ranging from determining the hygienic status of stainless steel \citep{jullien_bénézech_carpentier_lebret_faille_2003} to detecting whether a road is driveable or not for an autonomous vehicle \citep{guo_gerasimov_poulton_2006}, \citep{bystrov_2016}. Surface classification is, if one does not limit one self to any sub category, a very general problem statement and has thus naturally been approached in many different ways. In this section a few different methods are introduced, with particular focus on radar-based work. 

\subsubsection{Optical}

\subsubsection{Probing}

\subsubsection{Low frequency radar}

It is in many applications of interest to obtain images of subsurface identification nondestructively. In recent decades, ground penetrating radar (GPR) has found increasing use in evaluation of road conditions \citep{solla_gonzález-jorge_varela_lorenzo_2013} for preservation and maintenance of infrastructure.


A major challenge in radar sensing is to not only detect, but also to identify radar targets. This can for example be used for monitoring of urban environments \citep{harter_kowalewski_sit_jalilvand_ziroff_zwick_2014}.


Localization is the classic use-case for radars. 
\\ \\
 Radar  can detect relatively small targets at near or far distances and can measure their range with  precision  in  all weather,  which is  its chief  advantage when compared with other  sensors \citep{skolnik_2009}
\\ \\
Radar classification: 


% Classifying underground objects - this is a good paper we should look deeper into
Using a ground penetrating radar (GPR), it is possible to identify subsurface objects. 
Significantly lower frequency (900 MHz)
\citep{lu_pu_liu_2014}.
% Paper on ground penetrating radar
\citep{daniels_2004}

% RadarCat

\subsubsection{High frequency radar}
The RadarCat project is perhaps the work most closely related to this report \citep{yeo_2016}. As a part of Google's project Soli, RadarCat used Infineons radar sensor for accurate classification of materials. Detections was performed by placing an object of interest in direct contact with the sensor and determining the type by using a support vector machine. However, a central part of this approach is having the object in direct contact in a stationary setting, as opposed to at a range and in motion. The results found are however encouraging in the sense that high frequency radars are, at the very least to some degree, capable of classifying materials based solely on their radar response. 


\iffalse
 of urban environments and automotive radar 
\\ \\
Applications involving millimetre-wave radars and surface recognition is scarce.  
\\ \\
These radars commonly have wavelenghts of something something
\\ \\
If we however want to classify surface materials using a much shorter wavelength, the task changes dramatically as apsorption is near-instant. 
\\ \\
Something about radar wavelenght categories
\\ \\
Furthermore, radar sensors are commonly used in an array setting which permits beamforming and extraction of spatial information. 
\\ \\
If one on the other hand only has a singular radar sensor this is not possible and you must resort to other means.
\\ \\
Probably want something about IQ demodulation somewhere in here.
\fi
\subsection{Issues with previous work}
As seen above, surface classification has been done using optical methods as well as probing. An optical method based on camera images requires, besides the obvious need of a good enough camera, a specific illumination setup in order to work. This requirement could potentially limit the applications of an outdoor surface classifier. Using a radar based solution instead, we can perform surface classification without having to rely on ambient light, nor having to emit any visible light.

As of the probing approach, it has actually been proven that probing can be used successfully in surface identification of rough terrain in \citep{giguere_dudek_2011}. Here, an accelerometer is attached at the end of a small metal rod. The probe measures acceleration patterns that are later used for classification. However, with an instrument that has direct contact with the ground, the chances that the instrument breaks, or that it drags other objects along with it increases significantly. Hence our radar based approach has an advantage over probing in this remark.

\section{Thesis outline}

In chapter 2 we present an overview of a general radar system and introduce some crucial concepts for understanding the radar output used in this work. In chapter 3 we proceed by presenting our measurement setup and comment on the data collecting process and the data itself. Chapter 4 revolves around discussing ways of preprocessing the data and extracting relevant features from it. After the preprocessing, in chapter 5, we go through different classification schemes we have tested. The classification results are presented in a table for easy comparison. Finally, we conclude with an overall discussion of the work in chapter 6 and summarize our conclusions in chapter 7.

%Chapter 3: Mention something about the data collecting process. Motive our choice of parameters such as sampling frequency. Go through visual observations of our data.

%Chapter 4: Feature selection and preprocessing is considered. Signal structure is discussed primarily on an intuitive level. 

%Chapter 5: Classification schemes. Moving from simple to advanced, some classifiers of special interest are investigated. Results with regards to these classifiers are presented.

%Chapter 6: Disucssion.

%Chapter 7: Conclusion.