\chapter{Conclusions and Future Work}

\section{Conclusions}
In this work, the possibilities of using millimeter-wave radar for surface classification were explored. We found that by manually extracting features exploiting temporal behaviour and combining this with a deep neural network model, we were able to perform binary classification, distinguishing grass-surfaces from other types of surfaces with an accuracy as high as 98.5 \%. Using a median filter, predictions were improved, giving a perfect result on test data, as well as a promising result on basic real world applications. When confronted with unfamiliar surfaces, the model falls short, and further investigation in this area is required. 
% Mention that patent application has been submitted. Ask Peter how this ought to be phrased and what to include?


\section{Future work}
As debated in section \ref{disc_feat}, a \gls{dnn}'s capabilities of automatic feature extraction is worth evaluating to further improve performance. Another model related topic to explore, briefly metioned in the same section, is whether a linear model can be adapted to solve the problem as good as the \gls{dnn}.

Gathering more data, and in particular a more diverse dataset could also improve the model as a lacking aspect revolves around confusion when faced with unfamiliar surfaces. Regarding new surfaces, an intriguing, yet natural, extension of the model is to apply it to other tasks such as road condition monitoring or various tasks for indoor surfaces, including floors and carpets.


%	No feature extraction
%	Automatic feature extraction(/selection?)
%	More data! And more types of data
%	Test for other tasks - such as road condition monitoring
%	Implement velocity-dependent sampling rate
%	Linear and non-linear models
%	Outlier removal
