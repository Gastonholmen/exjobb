\chapter{Discussion}


\section{Moisture}

One particularly challenging aspect of adequately classifying surfaces is the ever-changing environmental conditions surronding the sensor. Of particular interest is the moisture content in the surfaces of interest. Greater soil moisture implies higher dielectric constant, which in turn increases radar wave scattering \citep{rappaport_2006}. Thus a single surface may very well change its scattering properties over time. 

% More things that make selecting data tricky

\section{Surface variances}

Such effects is difficult to account for when selecting data. Gather a dataset as diverse as possible

\section{Individual sensor performance}
Despite the fact that the radar sensors have high precision, the output can vary a lot from sensor to sensor. In particular, one sensor can have a significantly higher gain than another, meaning that two sensors can have a different measure of reflexivity on the very same surface. This problem can be dealt with in several ways. 

Mention what we did, and discuss pros and cons.

1. Choose features that are not affected by the issue. Normalize sweeps etc.
	A. Normalize sweeps so that the main peak is at height 1
	B. Normalize sweeps by dividing with mean 
