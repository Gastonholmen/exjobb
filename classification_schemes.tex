\chapter{Classification schemes}

\section{LSTM}
Mention CNN and LSTM being used for time series classification. Then bring up the article which combines these two to motivate this approach.

Observing one range bin at a time, one could think of the radar data as a time series. This motivates the use of some classification scheme that exploits temporal behaviour. Recurrent neral networks (RNNs) feature this by having feedback within individual layers in the network. \citep{karim_majumdar_darabi_chen_2018} The problem with RNNs, however, is that they suffer from a vanishing or exploding gradient, and can only sustain a short term memory. A way to combat this is to use a neural network layer called long short term memory (LSTM). These are thoroughly described in, for example \citep{hochreiter_schmidhuber_1997}

LSTM-layers have previously been used successfully for classifications in radar applications. For instance in \citep{jithesh_sagayaraj_srinivasa_2018} the method was able to classify flying objects from $\textbf{nnn}$ different classes with an accuracy of ...? In \citep{karim_majumdar_darabi_chen_2018}, the LSTM layer is used in combination with a fully convolutional neural network (FCN), which proves to be a significant improvement from just using FCNs when classifying time series.