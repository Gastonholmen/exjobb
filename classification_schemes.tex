\chapter{Classification schemes}

In this chapter a few classification models are evaluated for the feature extraction processes described in the preceding chapter. Detailed descriptions of each classification framework is omitted, but good sources for further explanation will be supplied for the interested reader. The aim of this chapter is to compare various models and to determine the most promising one with regards to accuracy, size and complexity.


\section{Evaluated models}

In this section the effectiveness of a few different models are compared. Two linear models and two non-linear models are evaluated. The two linear models considered, a \emph{Linear Discriminant Analysis} (LDA) and a \emph{Support Vector Machine} (SVM) model.

\subsection{Linear models}

\subsubsection{Linear Discriminant Analysis}

Linear Discriminant Analysis, or LDA for short, 



\subsection{Artificial neural networks}

Artificial neural networks (ANNs) constitute a class of nonlinear models designed to mimic biological neural systems. ANNs consist of multiple layers of neurons. 


ANNs have been widely applied to solve a great number of difficult problems in different areas, including pattern recognition, signal processing and language learning. 


\subsubsection{Feedforward neural networks}
Feedforward networks consist of an input layer, some hidden layers and an output layer. Each layer receives a set of input values $\mathbf{x}$ from the previous layer, multiplies the inputs by a set of weights $\mathbf{\beta}_{i}$ and applies a nonlinear activation function $F$ to the multiplication output. This output is then used as inputs for the sequent layer. The output of layer $i$ at neuron index $j$, $y_{i,j}$, is thus related to the previous layer output $\{{y}_{i-1, k}\}_{k=1}^{K}$ through

\begin{equation}
	y_{i,j} = F\Big\{\sum_{k=1}^{K}\beta_{i,j,k}y_{i-1,k}\Big\}
\end{equation}




Deep neural networks, or DNN for short, are neural networks that has more than one layer of hidden units between its inputs and outputs \citep{hinton_deng_yu_dahl_mohamed_jaitly_senior_vanhoucke_nguyen_sainath_2012}. 


DNN's in various forms have gained immense popularity over the past two decades achieving considerable success within a wide spectrum of applications, such as in image recognition \citep{szegedy_liu_jia_sermanet_reed_anguelov_erhan_vanhoucke_rabinovich_2018}, acoustic modeling of speech \citep{hinton_deng_yu_dahl_mohamed_jaitly_senior_vanhoucke_nguyen_sainath_2012}, 

\subsubsection{LSTM}

In previous models the data went through a feature extraction process before going into the model training. For this model, on the other hand, no feature extraction is performed. Instead, several consecutive sweeps of unprocessed IQ-data are used as input. Each range bin can be regarded as a one-dimensional time series. This motivates the use of some classification scheme that exploits temporal behaviour. Recurrent neral networks (RNNs) feature this by having feedback within individual layers in the network. \citep{karim_majumdar_darabi_chen_2018} The problem with RNNs, however, is that they suffer from a vanishing or exploding gradient, and can only sustain a short term memory. A way to combat this is to use a neural network layer called long short term memory (LSTM).

LSTM-layers have previously been used successfully for classifications in radar applications. For instance in \citep{jithesh_sagayaraj_srinivasa_2018} the method was able to classify flying objects from $\textbf{nnn}$ different classes with an accuracy of ...? The theory behind these layers are thoroughly described in, for example \citep{hochreiter_schmidhuber_1997}

Mention some source(s) using convolutional layers for radar classification.

 In \citep{karim_majumdar_darabi_chen_2018}, the LSTM layer is used in combination with a fully convolutional neural network (FCN), which proves to be a significant improvement from just using FCNs when classifying time series.

Describe our model.

% Define block styles
\tikzstyle{decision} = [diamond, draw, fill=blue!20, 
    text width=7.5em, text badly centered, inner sep=0pt]
\tikzstyle{block} = [rectangle, draw, fill=blue!20, 
    text width=7em, text centered, rounded corners, minimum height=4em]
\tikzstyle{line} = [draw, -latex']
\tikzstyle{cloud} = [draw, ellipse,fill=red!20, node distance=3cm,
    minimum height=4em]
\tikzstyle{blockgreen} = [rectangle, draw, fill=green!20, 
    text width=7em, text centered, rounded corners, minimum height=4em]
\tikzstyle{blockbrown} = [rectangle, draw, fill=brown!20, 
    text width=7em, text centered, rounded corners, minimum height=4em]

\begin{figure}
\centering
\begin{tikzpicture}[node distance = 3cm, auto]
	% Place nodes
	\node [blockbrown] (obtained) {Obtained data};
	\node [blockgreen, below left of=obtained] (training) {Training data};
	\node [block, below right of=obtained] (test) {Test data};
	\node [blockgreen, below of=training, node distance=2.2cm] (train pre) {Preprocess and extract features};
	\node [block, below of=test, node distance=2.2cm] (test pre) {Preprocess and extract features};
	\node [blockgreen, below right of=train pre] (est) {Estimate $\mathbf{\mu}_f$, $\mathbf{\sigma}_f$};
	\node [blockgreen, below left of=est] (scale train) {Scale to ZMUV};
	\node [block, below right of=est] (scale test) {Scale to ZMUV};
	\node [cloud, below right of=scale train] (train model) {Train model};
	\node [decision, below of=train model] (classify) {Classification};
	\node [block, below of=classify] (post) {Postprocessing};
	\node [block, below of=post, node distance=2.2cm] (output) {Model output};
	% Draw lines
	\path [line] (obtained) -| (training);
	\path [line] (obtained) -| (test);
	\path [line] (training) -- (train pre);
	\path [line] (test) -- (test pre);
	\path [line, dashed] (train pre) -| (est);
	\path [line,dashed] (est) -- (scale train);
	\path [line,dashed] (est) -- (scale test);
	\path [line] (test pre) -- (scale test);
	\path [line] (train pre) -- (scale train);
	\path [line] (scale train) |- (train model);
	\path [line] (train model) -- (classify);
	\path [line] (scale test) |- (classify);
	\path [line] (classify) -- (post);
	\path [line] (post) -- (output);
\end{tikzpicture}
\caption{Flowchart of classification scheme.}
\end{figure}

\section{Model evaluations}

\rowcolors{2}{gray!25}{white}
\begin{table}
	\begin{center}
\begin{adjustbox}{totalheight=\textheight-2\baselineskip}
		\begin{tabular}{|l|l|l|l|l|l|l|}
		\hline
		\rowcolor{gray!150}
		\rule{0pt}{25pt}\color{white}\textbf{Material} & \color{white}\textbf{LR} & \color{white}\textbf{RF} & \color{white}\textbf{\shortstack{LSTM \\CNN}} & \color{white}\textbf{SVM} & \color{white}\textbf{LDA} & \color{white}\textbf{DNN}\\
		pgrass1 & 98.7 & 98.4 & \cellcolor{red!20}94.3 & 97.95 & 97.75 & 98.25\\
		pgrass3 & 99.9 & 99.6 & 99.9 & 99.25 & 100.0 & 100.0\\
		pgrass4 & \cellcolor{red!20}94.05 & \cellcolor{red!20}92.25 & 97.7 & \cellcolor{red!20}93.25 & \cellcolor{red!20}91.95 & \cellcolor{red!20}94.7\\
		pgrass5 & 95.65 & 96.9 & 100.0 & \cellcolor{red!20}92.85 & \cellcolor{red!20}93.55 & 97.9\\
		qgrass1 & 97.5 & 97.45 & 97.3 & 96.15 & 97.25 & 95.75\\
		qgrass2 & 97.95 & 99.0 & 99.3 & 96.65 & 97.9 & 98.4\\
		qgrass3 & 99.95 & 100.0 & 99.8 & 99.95 & 100.0 & 99.95\\
		hgrass12 & 96.9 & 96.7 & 98.1 & 96.4 & \cellcolor{red!20}92.85 & 97.25\\
		hgrass2 & 97.9 & 97.7 & 99.7 & 97.85 & 96.45 & 98.4\\
		hgrass3 & 99.3 & 98.6 & 100.0 & 98.95 & 97.85 & 99.65\\
		hgrass4 & 99.45 & 98.85 & 99.9 & 99.1 & 97.6 & 99.85\\
		hgrass5 & 99.6 & 99.5 & 99.9 & 99.6 & 99.2 & 99.5\\
		hgrass6 & 100.0 & 100.0 & 100.0 & 100.0 & 99.85 & 100.0\\
		hgrass7 & 95.9 & 96.65 & 98.0 & \cellcolor{red!20}94.8 & \cellcolor{red!20}89.1 & 98.5\\
		hgrass8 & 97.8 & 95.65 & 100.0 & 95.15 & \cellcolor{red!20}89.5 & 97.5\\
		hgrass9 & 97.5 & 98.0 & 100.0 & 96.4 & \cellcolor{red!20}94.35 & 98.65\\
		hgrass10 & \cellcolor{red!20}94.95 & \cellcolor{red!20}93.0 & 98.5 & \cellcolor{red!20}94.1 & \cellcolor{red!20}89.4 & 95.25\\
		hgrass11 & 96.7 & \cellcolor{red!20}94.6 & 98.1 & 95.9 & \cellcolor{red!20}93.55 & 96.55\\
		pasph1 & 100.0 & 100.0 & 100.0 & 100.0 & 100.0 & 100.0\\
		pgravel1 & 98.05 & 99.65 & \cellcolor{red!20}72.0 & 99.4 & 97.8 & 98.15\\
		pasph2 & 99.95 & 100.0 & 99.5 & 99.95 & 99.95 & 99.95\\
		pasph3 & 100.0 & 100.0 & 100.0 & 100.0 & 99.95 & 99.9\\
		psoil1 & 99.75 & 99.85 & 99.9 & 99.9 & 99.65 & 99.9\\
		psoil2 & 98.3 & 99.5 & 99.0 & 99.6 & 98.65 & 99.15\\
		ptiles1 & 99.0 & 99.3 & 99.8 & 99.55 & 99.05 & 97.8\\
		qasph1 & 99.9 & 99.85 & 100.0 & 99.9 & 100.0 & 99.95\\
		qasph2 & 100.0 & 100.0 & 100.0 & 100.0 & 100.0 & 100.0\\
		qgravel1 & \cellcolor{red!20}88.35 & 95.4 & \cellcolor{red!20}76.7 & \cellcolor{red!20}93.25 & \cellcolor{red!20}91.5 & 97.15\\
		qsoil1 & 99.9 & 99.75 & 100.0 & 100.0 & 99.9 & 99.9\\
		qsoil2 & 97.8 & 95.25 & \cellcolor{red!20}90.6 & 98.75 & 96.95 & \cellcolor{red!20}94.05\\
		qtiles1 & 99.95 & 99.45 & 99.9 & 99.95 & 99.65 & 99.55\\
		rtiles1 & 99.95 & 99.9 & 99.9 & 100.0 & 100.0 & 99.9\\
		rgravel1 & 99.85 & 99.5 & 99.9 & 99.95 & 99.95 & 99.35\\
		rgravel2 & 98.7 & 99.4 & 100.0 & 99.5 & 99.8 & 99.35\\
		rasp1 & 100.0 & 100.0 & 100.0 & 100.0 & 99.9 & 99.9\\
		rsoil1 & 95.4 & 95.8 & 99.5 & 96.5 & 96.35 & \cellcolor{red!20}93.45\\
		G2gravel2 & 99.75 & 99.75 & 99.7 & 99.95 & 100.0 & 99.65\\
		G2gravel1 & 99.75 & 99.55 & 100.0 & 99.95 & 99.9 & 99.8\\
		G2soil1 & 96.65 & \cellcolor{red!20}94.4 & 95.7 & 97.6 & 95.35 & \cellcolor{red!20}94.3\\
		G2tiles1 & 99.9 & 99.65 & 99.4 & 100.0 & 100.0 & 100.0\\
		hsoil1 & \cellcolor{red!20}92.95 & 95.55 & 97.3 & \cellcolor{red!20}93.25 & \cellcolor{red!20}92.4 & \cellcolor{red!20}93.4\\
		hsoil2 & 95.25 & 95.1 & \cellcolor{red!20}92.3 & 95.1 & \cellcolor{red!20}92.9 & 95.5\\
		\hline
		\textbf{Mean} & 98.07 & 98.08 & 97.66 & 98.01 & 97.09 & \cellcolor{green!20}98.24\\
		\textbf{Median} & 98.85 & 99.35 & \cellcolor{green!20}99.8 & 99.32 & 98.28 & 99.25\\
		\textbf{Variance} & 5.8 & 4.67 & 31.69 & 5.51 & 11.66 & \cellcolor{green!20}4.15\\
		\hline
		\end{tabular}
\end{adjustbox}
	\end{center}
	\label{tab:loo}
	\caption{Leave-one-out accuracies for all collected data series and several different models.}
\end{table}

\subsection{Evaluation metrics}
Model evaluation is an important aspect in creating machine learning models. By using a bad evaluation strategy, one might construct a model that is seemingly good, but turns out to be useless in reality. There are several things to keep in mind when testing a model's performance. One of these is to use evaluation metrics that are relevant to the type of model that is being tested. For a classification model, a common metric is accuracy, which reveals the ratio between correct predictions and total predictions. A more informative metric is the confusion matrix which, in addition, provides details about the model's mispredictions. Two additional metrics, suitable for classification are log-loss and AUC \citep{zheng_2015}. 

However, selecting a suitable metric is not enough. Choosing what data to test your model on is equally important. By predicting on data that the model has been trained on, one could expect a very high accuracy. This accuracy, however, is not interesting at this point as a model is intended to be used on new, unseen data. For this reason, the dataset is split up in a training set and a test set in one of many ways \citep{raschka}.


\subsubsection{Validation set accuracy}
Splitting a dataset into a training and validation set often means selecting samples for training and validation completely randomly. This can be a good way of comparing different models' performances, but it is not the ideal method for a final model evaluation. While it is true that the validation set consists of samples that the model has never seen before, it \textit{has} trained on samples close to the validation samples due to the random selection. Oftentimes closely spaced samples can have a high resemblance, and even if the model has not trained on the validation samples, it achieves a high validation accuracy because it has trained on very similar samples.


\subsubsection{Leave-one-out accuracy}
The leave-one-out strategy means dividing the dataset into $n$ parts. In this work the data is split up in the $n=42$ different measured surfaces. The model validation is then performed in $n$ stages. For each stage, the model is trained on all data, except for one of the $n$ parts. After the training, the model makes predictions of the unseen data, and the accuracy is noted. When having a small dataset - or as in this case, an arguably small amount of surfaces that have been recorded - this method is useful in that it does not require us to withhold data from the model training \citep{raschka}. It also mimics a real scenario where the model predicts on a surface it has never seen before.

The result of the leave-one-out test is shown in table \ref{tab:loo}.










