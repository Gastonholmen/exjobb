\newenvironment{abstract}%
    {\cleardoublepage\thispagestyle{empty}\null\vfill\begin{center}%
    \bfseries{\textsf{Abstract}}\end{center}}%
    {\vfill\null}
\begin{abstract}
Classification of surfaces in the near field using millimeter-wave radar commonly considers the use of polarization based methods for road condition monitoring. When a surface consists of larger structures one instead wishes to monitor the surface topography. Analysis of scattering from rough surfaces is highly complex and relies on prior knowledge of surface structure. In this work a device moving at constant velocity is considered. By constructing a set of slow and fast time based features a machine learning classifier is used to distinguish grass target surfaces from asphalt, gravel, soil and tiled surfaces. It is found that using estimated autocovariances and average envelope shapes makes for an efficient feature extraction and that a small fully connected neural network classifier adequately manages to determine the surface type. The found model is accurate yet parsimonious and could be implemented with limited hardware requirements. Application of a median filter onto the sequence of classifier predictions effectively suppresses outlying predictions. This model can find use in autonomous devices that have tasks performed on designated surface types, such as in autonomous lawn mowers. 
	
%or statistical modeling of backscattering radiation from the surface one wishes to classify. This work considers the case of classifying grass surfaces from non-grass surfaces where the assumptions made in the aforementioned methods cannot be applied. Instead, we construct a set of slow time and fast time based features and use machine learning classifiers to distinguish grass from asphalt, gravel, soil and tiled surfaces. It is found that a autocovariance-based feature extraction process and a small neural net classifier adequately manages to determine the surface type, making for an effective yet parsimonious classifier that could be implemented with limited hardware requirements. Application of a median filter to the sequence of classifier prediction confidences effectively suppresses outlying predictions. This model can find use in moving devices that have a task performed on a designated surface type, such as autonomous lawn mowers or autonomous vacuum cleaners. 

%Millimeter-wave radars have received increasing attention in consumer electronics. In this report the use of a 60 GHz pulsed coherent radar for surface classification is evaluated for a small device moving at constant velocity. This work is focused on determining if a target surface is grass covered or belonging to a set of 5 other surface types commonly found adjacent to lawns. It is found that the use of temporal features and a neural network classifier allows for accurate binary classification for the chosen targets. 

\end{abstract}
