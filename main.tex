\documentclass[a4paper, 12pt]{article}

\usepackage{natbib}
\usepackage{amsmath}
\usepackage{amssymb}
\usepackage{amsfonts}
%\usepackage{notoccite}

% Expectation symbol
\DeclareMathOperator*{\E}{\mathbb{E}}

\begin{document}





\section{Introduction}

This paper presents an algorithm for surface classification using millimeter-wave radar. 

Fundamentally, radar is based on the concept of transmitting and recieving electromagnetic radiation. Radar response depends on the surrounding region of the sensor as objects in its vicinity scatter transmitted waves differently \citep{richards_2014}.  

 As technolgy is becoming increasingly intertwined with everyday life a multitude of new areas for smart sensing has opened up. One sensor category that has been recieving much attention is radars with very high frequency. These sensors combine a multitude of desirable features packed into a favorable form factor. These capabilities has made high frequency radars an increasingly popular choice in a steadily growing number of applications, such as monitoring of vital signs \citep{kuo_lin_yu_lo_lyu_chou_chuang_2016}, gesture recognition \citep{lien_gillian_karagozler_amihood_schwesig_olson_raja_poupyrev_2016}

Alongside with this development another trend within data science has emerged; the movement towards machine learning centric methods . This transition, spanned over multiple decades, has placed machine learning as one of the main-stays of information technology and a rather central part of our life. With ever increasing amounts of data, smart data analysis has proven key for technological progress \citep{a_smola_svn_vishwanathan_2010}.

Machine learning algorithms fundamentally utilizes various statistical techniques in order to allow computer systems to progressively improve performance. Data scientists have found this subclass of aritficial intelligence extremely practical, especially for data that lack an easily predictable structure. 

The amount of millimetre-wave radar applications is growing steadily.

With a continuously increasing demand for small sensors for use in a world where 

\subsection{Motivation}

Classification of surfaces is an important task. It appears in many areas.
\\ \\
You can determine surfaces in many different ways - optical, probing etc.
\\ \\
\noindent Optical: Images has been successful in determining surface roughness. One such method involves images of three-dimensional surface textures under different directions of illumination, and after some image processing techniques using a support vector machine for classification \citep{dong_duan_yang_2008}. 
\\ \\
\noindent Probing: One may also measure directly onto a surface directly through a probing approach. By sweeping a thin film across a fabric with constant contact force a depth-dependant charge is induced. After reading the output charge over some time interval it is then possible to extract information for detecting the texture of the fabric \citep{song_han_hu_li_2014}. A similar approach was used in \citep{strese_schuwerk_iepure_steinbach_2017}, where again surface classification was performed through direct contact. In this case features where instead extracted from the sound produced by vibrations generated by the movement. 
\\ \\
It is often desirable to to this without direct contact i.e. probing
\\ \\
Localization is the classic use-case for radars. 
\\ \\
Radar classification: Previous work and how they are different: A major challenge in radar sensing is to not only detect, but also to identify radar targets. This can for example be used for monitoring of urban environments with digital beamforming \citep{harter_kowalewski_sit_jalilvand_ziroff_zwick_2014}, or with a radar system mounted on a rotary table unit. 

 of urban environments and automotive radar 
\\ \\
Applications involving millimetre-wave radars and surface recognition is scarce.  
\\ \\
These radars commonly have wavelenghts of something something
\\ \\
If we however want to classify surface materials using a much shorter wavelength, the task changes dramatically as apsorption is near-instant. 
\\ \\
Something about radar wavelenght categories
\\ \\
Furthermore, radar sensors are commonly used in an array setting which permits beamforming and extraction of spatial information. 
\\ \\
If one on the other hand only has a singular radar sensor this is not possible and you must resort to other means.
\\ \\
Probably want something about IQ demodulation somewhere in here.

\subsection{Problem formulation}

We develop a pipeline for effectively solving this problem. 
\\ \\
We present modelling options with high accuracy and efficacy. 
\\ \\ 
We present results based on a 60GHz Acconeer micro radar. 
\\ \\
In this paper we introduce means of surface identification based on the output from a single 60 GHz radar sensor moving across a surface with constant height. 

\subsection{Thesis outline}

Chapter 2: We present a radar systems overview and introduce some crucial concepts for understanding the radar output used in this work. 
\\ \\
Chapter 3: Mention something about the data collecting process. Motive our choice of parameters such as sampling frequency. Go through visual observations of our data.
\\ \\
Chapter 4: Feature selection and preprocessing is considered. Signal structure is discussed primarily on an intuitive level. 
\\ \\
Chapter 5: Classification schemes. Moving from simple to advanced, some classifiers of special interest are investigated. Results with regards to these classifiers are presented.
\\ \\
Chapter 6: Disucssion.
\\ \\
Chapter 7: Conclusion.

\section{Chapter 2: Radar system overview}

In this chapter we introduce three fundamental radar concepts which will be important for understanding future feature selection discussions. 
\\ \\
The radar equation.
\\ \\
The matched filter.
\\ \\
The IQ demodulation scheme. 
\\ \\
The radar system used for this project is a 60 GHz radar developed by Acconeer AB.

\subsection{The radar principle}

The radar principle is at its core simple.  A wavelet pulse $x_T(t)$ with some carrier frequency $\Omega$  is transmitted towards an object of interest. After some short time $t_1$, the radar listens for an echo. If no echo is received, it means there is no object present at distance\footnote{The reason for dividing by 2 is that the wavelet pulse must first travel from the radar to the object, and then find its way back to the radar.}
\begin{equation}
	d_1 = \frac{v_0\cdot t_1}2
\end{equation}
where $v_0$ is the speed of an electromagnetic wave in air. After the previously transmitted wavelet is guaranteed to have died out, a new one is transmitted. Again, the radar listens for an echo, but this time after another time $t_2$ has passed. $t_2$ is greater than $t_1$, meaning the radar listens for echoes further away. If an echo is received, it means there is an object present at distance 
$
	d_2 = \frac12(v_0\cdot t_2).
$
This process is repeated for different time delays $t_i$. The chosen time delays can be adjusted depending on in what ranges one wants to search in, and how good range resolution one seeks.

Together, the echoes (or lack thereof) from each transmitted wavelet make up one sweep. A sweep is often plotted in an amplitude-vs-range, or amplitude-vs-depth diagram as in figure (...). This plot is a good way to visualize at what ranges objects are present.

* Plot of sweep *

\subsection{Matched filter}

something something desired frequency response of the recieving antenna. 

In any radar system a good Signal-to-Noise Ratio (SNR) is a highly desired property. Finding a reciever frequency response which maximizes SNR is thus an important topic. Denoting the reciever output as $y(t)$ and the incoming waveform as $x(t)$ the output spectrum will be a convolution of $x(t)$ and the system impulse response $h(t)$, or conversely a multiplication in the frequency domain $Y(\Omega) = X(\Omega)H(\Omega)$. If we seek to maximize SNR at some arbitrary point in time $T_M$ the power at that very instant is
%
\begin{equation}
	|y(T_M)|^{2} = |\frac{1}{2\pi}\int X(\Omega)H(\Omega)e^{j\Omega T_M} d\Omega|^{2}.
\end{equation}

Now we consider interference in the form of spectrally flat noise with power spectral density $\sigma^2$ W/Hz. The SNR $\xi$ measured at time $T_M$ can then be described as the ratio between the total signal power and the total noise power
%
\begin{equation}
	\xi
	= \frac{|y(T_M)|^{2}}{(1/2\pi)\int|\sigma H(\Omega)|^{2}d\Omega}
	= \frac{|(1/2\pi)\int X(\Omega)H(\Omega)e^{j\Omega T_M}d\Omega|^2}{(\sigma^2/2\pi)\int|H(\Omega)|^{2}d\Omega}
\end{equation}
%
which clearly depends on which reciever response is used. It can from above expression be shown [reference] that the maximum $\xi$ is obtained when 
%
\begin{gather}
 H(\Omega) = \alpha X*(\Omega)e^{j\Omega T_M}, \text{ or} \\
\label{eq:123}
h(t) = \alpha x^*(T_M - t)
\end{gather}
%
where $\alpha$ is an arbitrary constant which has no impact on the resulting SNR. Examining $h(t)$ above we see that the optimal filter for maximizing SNR is when the coefficients consist of the transmitted waveform conjugated and time-reversed. This filter is called a \emph{matched filter} due to the symmetrical relationship between waveform and impulse response.

One way of interpreting the matched filter is by viewing the filtering as a correlation. If we denote $\bar{x}(t)$ as the sum of both target and noise components the output $y(t)$ is given by
%
\begin{equation}
	y(t) = \int \bar{x}(s)h(t-s)ds = \alpha\int \bar{x}(s)x^*(s + T_M - t)ds
\end{equation}
%
which is recognized as the cross-correlation between noisy signal $\bar	{x}(t)$ and transmitted waveform $x(t)$ evaluated at lag $T_M-t$. By shifting the constant lag $T_M$ we then can obtain the full cross-correlation 

\subsection{IQ demodulation}
A common type of data when working with radar signal processing is IQ-data (in- and quadrature phase).  This type of data is useful in that it contains information not only about amplitude, but about the phase of the radar signal as well \citep{richards_2014}. In particular, this is good for situations where small changes in the radar signal need to be detected, (Mention a case such as recording over grass...) as the phase is more sensitive to changes than the amplitude \citep{lien_gillian_karagozler_amihood_schwesig_olson_raja_poupyrev_2016}. If an object is detected at distance $r(t_1)$ from the radar at time $t_1$, and shortly thereafter, the object is moved to distance $r(t_2)$, the corresponding difference in phase will be
\begin{equation}
	\label{eq:phase_diff}
	\Delta\phi(t_1, t_2)=\frac{4\pi}{\lambda}(r(t_2)-r(t_1)) \quad\quad \textrm{mod 2$\pi$}
\end{equation}
By having a sampling frequency which is too low, the range difference in $\eqref{eq:phase_diff}$ could potentially become very big, and the phase difference would fluctuate over time and be incomprehensible.

In the case of material classification, a high sampling frequency is highly beneficial. When moving across surfaces characterized by tiny details, a high sampling frequency is essential in capturing the shape of these.

IQ-data is derived through a process commonly known as IQ- or quadrature demodulation. Raw data, received from the radar, is split up into two channels, according to figure (...). These two channels make up a real part in-phase (I) and quadrature phase (Q) part. From these we can obtain abs and phase...

\section{Data}
\subsection{Data Collecting}
Having reliable data is a fundamental requirement in building a good model. Hence, choosing suitable parameters such as sampling frequency, regions of interest and planning a good measurment setup are all crucial tasks. The sampling frequency is particularly important to consider when working with means that could potentially cause aliasing. One such case is the DFT \textbf{[SOURCE: some book in my bookshelf]}. If the sampling frequency is too low, we get aliasing etc. etc. 

In order to assure that aliasing is avoided, the maximal frequency component registered by the radar must be surpassed by half the sampling frequency
\begin{equation}
	f_{max} < \frac{f_s}{2}.
\end{equation}
The sampling frequency can be chosen accordingly, assuming the maximal frequency component is known. 

In figure (...) a vehicle is moving forward with constant speed, having a sensor mounted at the front. As it moves forward, small objects on the ground get closer to the radar sensor with some speed. This gives rise to a doppler frequency being registered by the radar, which is \citep{lien_gillian_karagozler_amihood_schwesig_olson_raja_poupyrev_2016}
\begin{equation}
	f_{d} = \frac{2v_\perp}{\lambda}.
\end{equation}
...


\subsection{Chapter 2.5: Data exploration}

Good source on PCA
\cite{hyvasrinen_karhunen_oja_2004}
PCA 125-143

Information theory
105-122
Argument for downsampling in range: In an information theoretic framework we interpret a random variable by how unpredicable or unstructured an observation of the variable is. This concept, examining the randomness of a variable, is commonly measured through entropy. Directly related to entropy is mutual information which essentially is how much information each member of a set have on the other members. \cite{hyvasrinen_karhunen_oja_2004}. Ideally one wants measurements that have a low measure of mutual information, meaning that each datapoint contain information not found elsewhere in the set. 

Observing a typical radar sweep[ref to plot with sweep] we note that points are very closely related on a small range scale, and nearly identical if we were to examine them on a sample by sample basis. One could argue that the mutual information found in the set is very high and that the entropy in each datapoint is low. Hence, to lower the first and increase the latter, one could downsample by some factor $D$ in range without significant loss of information.


PCA 125-143: Through the point and feature selection methods described in previous sections we obtain high dimensional feature vectors. Getting an intuitive feel for such data extracted in these processes is difficult as direct plotting is limited to three dimensions. 

Principal Component Analysis (PCA) is  a classical technique in statistical data analysis which takes a large set of multivariate variables and finds a smaller set of variables with less redundancy. Critically, PCA finds a rotated orthogonal coordinate system such that the elements of the set become uncorrelated. Projecting elements on the principal axes corresponding to the directions of maximal variance a good approximation of the original data in lower dimension is obtained\citep{hyvasrinen_karhunen_oja_2004}. 

\section{Chapter 3: Feature selection}

In order to make sense of data one must impose some reasonable assumptions. 
\\ 
One idea is to view each measurement, taken at time instance $t$ at range $r$ for material $m$, as drawn from some 
complex distribution specific for that combination of $r$ and $s$ but independent of time of sampling $t$. 
\\
Yada yada there is two combatting ideas here

\subsection{Preprocessing}

\subsubsection{Downsampling}
Add reference to Information Theoretic part

Since the correlation between neighboring range samples is very high we can downsample without significant loss of information. The downsampling process essentially require three hyperparameters: A starting range $R_{start}$ and an end range $R_{end}$ where we begin and end our downsampling process respectively, as well as a downsampling factor $D$.

\begin{equation}
	x_D(n, t) = x(R_{start} + nD, t) \quad \text{for}\quad n=0...\frac{R_{end}-R_{start}}{D}
\end{equation}

\subsubsection{Sweep Normalization}
The gain sensitivity of the Acconeer radar chips can vary a lot from sensor to sensor. Makes it meaningless to measure, and classify on one sensor which is very different from the one that measured the training data. 

Explain normalization procedure: divide all sweeps within a feature box with the total energy within that feature box. 

Mention this was the best approach, but do not go into detail about the other methods - if so, just mention them briefly.



\subsubsection{Using multiple sweeps for feature selection}

In the radar system multiple sources of noise are introduced.
\\ \\
Noise sources: Thermal emissions of target, random currents in electrical components, data quantization, etc. etc. \citep{w_doerry_2016}
\\ \\
In the beginning section of this chapter we introduced two conflicting interpretations of the radar signal. In the first the incomming wavelet is drawn from a complex distribution, and each sweep is independent from the rest. In the second we instead considered that there may be some significant correlations in time to take into account; surface shapes and objects passing by the sensor are detected multiple times in a perhaps predictable and useful manner.  

The first suggest that signal averaging over a number of sweeps should yield a more and more accurate estimate. Thus the first approach greatly benefit from using multiple sweeps to find an average over some predefined number of sweeps.  In order to utilize the time correlations assumed in the second interpretation processing in time is of course demanded, and a number of sweeps need to be analyzed at once to extract features related to this way of viewing the radar response. 

Thus a predefined number is chosen as the number of sweeps used to extract each feature. While each feature may be extracted with higher accuracy, it is done at the price of a lower number of classifications per second. The relation between classification rate $F_c$, sampling speed $F_s$ and sweeps per feature $T_s$ is



\begin{equation}
	F_c = \frac{F_s}{T_s}
\end{equation}

\subsection{Features}

Fast time and slow time features 

Suggetsed general template of subsections: 
\begin{itemize}
	\item{Motivate the idea}
	\item{Present the actual relevance in terms of plots or other investigation results}
	\item{Show/explain the implementation}
\end{itemize}

\subsubsection{Average energy}
The average energy tells us how much energy is reflected back to the radar. Hence it can be regarded as a measure of how good of a reflector the underlying surface is. The energy depends on the shape of the surface, as well as its dielectric constant. Compared to other materials, grass has a very different surface shape, which potentially gives it a very different reflexivity. However, its dielectric constant could also vary a lot from sunny and rainy days, making it hard to guess its reflective properties.

By computing the average energy of single sweeps from different surfaces we obtain the results in figure (plots of sweep energies). Clearly, the grass surface has a much lower energy. $<$We have to compare different types of grass. Wet, dry, tall, short, etc.$>$

To get a more robust measure of the average energy we not only compute the average over selected range bins of a single sweep, but we average over a few consecutive sweeps as well. Mathematically, the feature we end up using becomes
\begin{equation}
	P(t_m) = \frac{1}{NT}\sum_{t=0}^{T-1}\sum_{n=0}^{K-1}x(n, t_m + t)x^*(n, t_m + t).
\end{equation}
Add some talk about this feature being left unused due to normalization.




\subsubsection{Expected signal}
The most simple feature would be to just use the plain sweeps as features. A simple next step could be to average a few sweeps in slow time, making the samples 

\begin{equation}
	s_i(t_m) = \E\{X_{i,t_m}\} = \frac{1}{T}\sum_{t=0}^{T-1}x(i, t_m + t)
\end{equation}


\subsubsection{Fourier Transform}
A "feature box" consisting of $T$ sweeps is selected. Viewing this as a matrix with elements $X_{n,t}$, where $n$ denotes slow time, and $t$ fast time we can write a Fourier transform in slow time at range $r$, within this box as
\begin{equation}
	\mathbb{X}_k^{(r)} = \sum_{n=0}^{T-1}X_{n,r}\exp\Big[-2\pi i\frac{nk}{T}\Big] \quad k=0, ..., T-1
\end{equation}
To avoid aliasing we refer to chapter earlier in report where this is discussed. Mention that $\mathbb{X}_k^{(r)}$, for all $r$, $k$, is flattened to a vector, resulting in one new sample.




\subsubsection{Variance}

Include a discussion about the biased/unbiased estimate \\

Assume $x(n,t)$ are samples from a complex normal distribution $X_{n,t}$. Variance $v_i(t_m)$ at range $i$ at time $t_m$ can then be approximated using

\begin{equation}
\label{eq:var}
\begin{gathered}
	v_i(t_m) = \E\{ (X_{i,t_m} - \E\{X_{i,t_m}\})^2\} \\
	= \frac{1}{T-1}\sum_{t=0}^{T-1}(x(i, t_m + t) - s_i(t_m))^*(x(i, t_m + t) -  s_i(t_m))
\end{gathered}
\end{equation}

\subsubsection{Autocovariance in slow time}

\textbf{Add some motivation for using autocovariance}

\noindent
For some stochastic process $P_t$ we can define the autocovariance $\gamma$ as

\begin{equation}
	\gamma(t, s) = \E\big\{(P_t - \mu_t)(P_s - \mu_s)\big\}
\end{equation}

If $X_t$ is a weakly stationary process the first moment (mean) and autocovariance do not vary over time.  The autocovariance then only depends on the difference between $s$ and $t$, making it possible to rewrite as

\begin{equation}
	\gamma(\tau) = \E\big\{(P_t - \mu)(P_{t+\tau} - \mu)\big\}
\end{equation}

Assuming weak stationarity over a few rangebins we can estimate the autocovariance in slow time for each selected range. 

\begin{equation}
\begin{gathered}
	\gamma_i(t_m, \tau) = \E\big\{(X_{i,t_m} - \E\{X_{i, t_m}\})^*(X_{i, t_m+\tau} - \E\{X_{i, t_m+\tau}\})\big\}\\
	= \frac{1}{T-1}\sum_{t=0}^{T-1-\tau}(x(i, t_m + t) - s_i(t_m))^*(x(i, t_m + t + \tau) - s_i(t_m + \tau))
\end{gathered}
\end{equation}•

We can drop the $\tau$ in the final expectation term, as $s_i(t)$ are estimated in batch from $t_m$ to  $t_m + T$ yielding

\begin{equation}
	\gamma_i(t_m, \tau) = \frac{1}{T-1}\sum_{t=0}^{T-1- \tau}(x(i, t_m + t) - s_i(t_m))^*(x(i, t_m + t + \tau) - s_i(t_m))
\end{equation}
as our expression for autocovariance. Note that the variance in \ref{eq:var} collapses to the autocovariance at 0 lag. 


\section{Chapter 4: Classification schemes}

\section{Chapter 5: Discussion}


\subsection{Moisture}

One particularly challenging aspect of adequately classifying surfaces is the ever-changing environmental conditions surronding the sensor. Of particular interest is the moisture content in the surfaces of interest. Greater soil moisture implies higher dielectric constant, which in turn increases radar wave scattering \citep{rappaport_2006}. Thus a single surface may very well change its scattering properties over time. 

% More things that make selecting data tricky

\subsection{Surface variances}

Such effects is difficult to account for when selecting data. Gather a dataset as diverse as possible

\subsection{Individual sensor performance}
Despite the fact that the radar sensors have high precision, the output can vary a lot from sensor to sensor. In particular, one sensor can have a significantly higher gain than another, meaning that two sensors can have a different measure of reflexivity on the very same surface. This problem can be dealt with in several ways. 

Mention what we did, and discuss pros and cons.

1. Choose features that are not affected by the issue. Normalize sweeps etc.
	A. Normalize sweeps so that the main peak is at height 1
	B. Normalize sweeps by dividing with mean 

\newpage
\bibliography{refs}
\bibliographystyle{apalike}


\end{document}
