\documentclass[a4paper, 12pt]{article}

\usepackage{amsmath}

\begin{document}


\section{Introduction}


\subsection{Motivation}

Classification of surfaces is an important task.

You can determine surfaces in many different ways - optical, probing etc.

It is often desirable to to this without direct contact i.e. probing

Radar has successfully been used to for example locate salt in the ground

These radars commonly have wavelenghts of something something

If we however want to classify surface materials using a much shorter wavelength, the task changes dramatically as apsorption is near-instant. 

Something about radar wavelenght categories

Furthermore, radar sensors are commonly used in an array setting which permits beamforming and extraction of spatial information. 

If one on the other hand only has a singular radar sensor this is not possible and you must resort to other means.

Probably want something about IQ demodulation somewhere in here.

\subsection{Problem formulation}

We develop a pipeline for effectively solving this problem. 

We present modelling options with high accuracy and efficacy. 

We present results based on a 60GHz Acconeer micro radar. 

In this paper we introduce means of surface identification based on the output from a single 60 GHz radar sensor moving across a surface with constant height. 

\subsection{Thesis outline}

Chapter 2: We present a radar systems overview and introduce some crucial concepts for understanding the radar output used in this work. 

Chapter 3: Feature selection and preprocessing is considered. Signal structure is discussed primarily on an intuitive level. 

Chapter 4: Classification schemes. Moving from simple to advanced, some classifiers of special interest are investigated. Results with regards to these classifiers are presented.

Chapter 5: Disucssion.

Chapter 6: Conclusion.

\section{Radar system overview}

In this chapter we introduce three fundamental radar concepts which will be important for understanding future feature selection discussions. 

The radar equation.

The matched filter.

The IQ demodulation scheme. 

The radar system used for this project is a 60 GHz radar developed by Acconeer AB.

An antenna transmits a wavelet signal towards an object of interest. After a brief period of time a second wavelet signal is generated and \emph{mixed} with data from a recieving antenna. This procedure is repeated, every time slightly delaying the generation of the second wavelet and thus mixing with a different section of the incoming pulse. 

Throgh this methodology we can effectively produce

\subsection{The radar principle}

The radar principle is at its core simple.  A wavelet pulse $x_T(t)$ with some carrier frequency $\Omega$  is transmitted towards an object of interest. 

etc etc..


\subsection{Matched filter}

something something desired frequency response of the recieving antenna. 

In any radar system a good Signal-to-Noise Ratio (SNR) is a highly desired property. Finding a reciever frequency response which maximizes SNR is thus an important topic. Denoting the reciever output as $y(t)$ and the incoming waveform as $x(t)$ the output spectrum will be a convolution of $x(t)$ and the system impulse response $h(t)$, or conversely a multiplication in the frequency domain $Y(\Omega) = X(\Omega)H(\Omega)$. If we seek to maximize SNR at some arbitrary point in time $T_M$ the power at that very instant is
%
\begin{equation}
	|y(T_M)|^{2} = |\frac{1}{2\pi}\int X(\Omega)H(\Omega)e^{j\Omega T_M} d\Omega|^{2}.
\end{equation}

Now we consider interference in the form of spectrally flat noise with power spectral density $\sigma^2$ W/Hz. The SNR $\xi$ measured at time $T_M$ can then be described as the ratio between the total signal power and the total noise power
%
\begin{equation}
	\xi
	= \frac{|y(T_M)|^{2}}{(1/2\pi)\int|\sigma H(\Omega)|^{2}d\Omega}
	= \frac{|(1/2\pi)\int X(\Omega)H(\Omega)e^{j\Omega T_M}d\Omega|^2}{(\sigma^2/2\pi)\int|H(\Omega)|^{2}d\Omega}
\end{equation}
%
which clearly depends on which reciever response is used. It can from above expression be shown [reference] that the maximum $\xi$ is obtained when 
%
\begin{gather}
 H(\Omega) = \alpha X*(\Omega)e^{j\Omega T_M}, \text{ or} \\
\label{eq:123}
h(t) = \alpha x^*(T_M - t)
\end{gather}
%
where $\alpha$ is an arbitrary constant which has no impact on the resulting SNR. Examining $h(t)$ above we see that the optimal filter for maximizing SNR is when the coefficients consist of the transmitted waveform conjugated and time-reversed. This filter is called a \emph{matched filter} due to the symmetrical relationship between waveform and impulse response.

One way of interpreting the matched filter is by viewing the filtering as a correlation. If we denote $\bar{x}(t)$ as the sum of both target and noise components the output $y(t)$ is given by
%
\begin{equation}
	y(t) = \int \bar{x}(s)h(t-s)ds = \alpha\int \bar{x}(s)x^*(s + T_M - t)ds
\end{equation}
%
which is recognized as the cross-correlation between noisy signal $\bar	{x}(t)$ and transmitted waveform $x(t)$ evaluated at lag $T_M-t$. By shifting the constant lag $T_M$ we then can obtain the full cross-correlation 

\subsection{IQ demodulation}

\section{Feature selection}

\section{Classification}

\section{Discussion}

\end{document}
