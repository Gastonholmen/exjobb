\documentclass[a4paper, 12pt]{article}

\usepackage{amsmath}

\begin{document}


\section{Introduction}

In this report a micro radar system is utilized for surface classification. Specifically, a radar sensor is placed on the inside of a robot facing downwards, with the objective of distinguishing if the surface below is made of grass and dirt or not.

\section{Radar system overview}

The radar system used for this project is a 60 GHz radar developed by Acconeer AB.

An antenna transmits a wavelet signal towards an object of interest. After a brief period of time a second wavelet signal is generated and \emph{mixed} with data from a recieving antenna. This procedure is repeated, every time slightly delaying the generation of the second wavelet and thus mixing with a different section of the incoming pulse. 

Throgh this methodology we can effectively produce

\subsection{The radar principle}

The radar principle is at its core simple.  A wavelet pulse $x_T(t)$ with some carrier frequency $\Omega$  is transmitted towards an object of interest. 

etc etc..


\subsection{Matched filter}

something something desired frequency response of the recieving antenna. 

In any radar system a good Signal-to-Noise Ratio (SNR) is a highly desired property. Finding a reciever frequency response which maximizes SNR is thus an important topic. Denoting the reciever output as $y(t)$ and the incoming waveform as $x(t)$ the output spectrum will be a convolution of $x(t)$ and the system impulse response $h(t)$, or conversely a multiplication in the frequency domain $Y(\Omega) = X(\Omega)H(\Omega)$. If we seek to maximize SNR at some arbitrary point in time $T_M$ the power at that very instant is
%
\begin{equation}
	|y(T_M)|^{2} = |\frac{1}{2\pi}\int X(\Omega)H(\Omega)e^{j\Omega T_M} d\Omega|^{2}.
\end{equation}

Now we consider interference in the form of spectrally flat noise with power spectral density $\sigma^2$ W/Hz. The SNR $\xi$ measured at time $T_M$ can then be described as the ratio between the total signal power and the total noise power
%
\begin{equation}
	\xi
	= \frac{|y(T_M)|^{2}}{(1/2\pi)\int|\sigma H(\Omega)|^{2}d\Omega}
	= \frac{|(1/2\pi)\int X(\Omega)H(\Omega)e^{j\Omega T_M}d\Omega|^2}{(\sigma^2/2\pi)\int|H(\Omega)|^{2}d\Omega}
\end{equation}
%
which clearly depends on which reciever response is used. It can from above expression be shown [reference] that the maximum $\xi$ is obtained when 
%
\begin{gather}
 H(\Omega) = \alpha X*(\Omega)e^{j\Omega T_M}, \text{ or} \\
\label{eq:123}
h(t) = \alpha x^*(T_M - t)
\end{gather}
%
where $\alpha$ is an arbitrary constant which has no impact on the resulting SNR. Examining $h(t)$ above we see that the optimal filter for maximizing SNR is when the coefficients consist of the transmitted waveform conjugated and time-reversed. This filter is called a \emph{matched filter} due to the symmetrical relationship between waveform and impulse response.

One way of interpreting the matched filter is by viewing the filtering as a correlation. If we denote $\bar{x}(t)$ as the sum of both target and noise components the output $y(t)$ is given by
%
\begin{equation}
	y(t) = \int \bar{x}(s)h(t-s)ds = \alpha\int \bar{x}(s)x^*(s + T_M - t)ds
\end{equation}
%
which is recognized as the cross-correlation between noisy signal $\bar	{x}(t)$ and transmitted waveform $x(t)$ evaluated at lag $T_M-t$. By shifting the constant lag $T_M$ we then can obtain the full cross-correlation 

\subsection{IQ demodulation}

\section{Feature selection}

\section{Classification}

\section{Discussion}

\end{document}
